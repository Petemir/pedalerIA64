\subsubsection{Linea comandos}
\label{subsec:cli}
Una vez compilado el \textbf{TP} (mediante el comando \textit{make}, pues el archivo Makefile se encuentra incluido), se puede ver la ayuda del programa ejecutando únicamente \textit{./main}. Por razones de completitud, se explica aquí también cómo utilizar el programa. \vspace{\baselineskip}

La estructura para aplicar un efecto a un archivo de audio es la siguiente:

\lstset{language=bash}
\begin{lstlisting}[frame=single]
./main INFILE OUTFILE EFFECT ARGS
\end{lstlisting}

\begin{itemize}
 \item \textbf{INFILE} es el archivo de audio de entrada, siempre en formato WAV.
 \item \textbf{OUTFILE} es el nombre deseado del archivo de salida, con extensión .WAV.
 \item \textbf{EFFECT} es un guión, seguido del caracter asociado al efecto a aplicar.
 \item \textbf{ARGS} son los argumentos dependientes del efecto definido en \textbf{EFFECT}.
\end{itemize}

La lista de los efectos y los rangos de los argumentos correspondientes (los mismos se explicarán en la sección \ref{sec:desarrollo} de Desarrollo de cada efecto) se pueden consultar en la siguiente tabla:\vspace{\baselineskip}

\begin{table}[H]
\centering
\begin{tabular}{|c|c|c|c|l|l|c|c|l|c|l|c|c|l|l|}
\hline
\multirow{2}{*}{{\bf Nombre}} & \multicolumn{2}{c|}{{\bf Caracter}} & \multicolumn{12}{c|}{\multirow{2}{*}{{\bf Argumentos}}}                                                                                                                                                                                                                                                                                                       \\ \cline{2-3}
                              & {\bf ASM}         & {\bf C}         & \multicolumn{12}{c|}{}                                                                                                                                                                                                                                                                                                                                        \\ \hline
Copiar                        & C                 & c               & \multicolumn{12}{c|}{Ninguno}                                                                                                                                                                                                                                                                                                                                 \\ \hline
Delay                         & D                 & d               & \multicolumn{6}{c|}{\begin{tabular}[c]{@{}c@{}}\underline{Delay}:\\ 0.0-5.0 segundos\end{tabular}}                                                                                & \multicolumn{6}{c|}{\begin{tabular}[c]{@{}c@{}}\underline{Decay}: \\ 0.00-1.00\end{tabular}}                                                                                                      \\ \hline
Flanger                       & F                 & f               & \multicolumn{4}{c|}{\begin{tabular}[c]{@{}c@{}}\underline{Delay}: \\ 0-15 milisegundos\end{tabular}}                   & \multicolumn{4}{c|}{\begin{tabular}[c]{@{}c@{}}\underline{Rate}: \\ 0.1-5.0 hertz\end{tabular}}                                   & \multicolumn{4}{c|}{\begin{tabular}[c]{@{}c@{}}Amp: \\ 0.65-0.75\end{tabular}}                                           \\ \hline
Vibrato                       & V                 & v               & \multicolumn{6}{c|}{\begin{tabular}[c]{@{}c@{}}\underline{Depth}: \\ 0-3 milisegundos\end{tabular}}                                                                               & \multicolumn{6}{c|}{\begin{tabular}[c]{@{}c@{}}\underline{Mod}: \\ 0.1-5.0 hertz\end{tabular}}                                                                                                    \\ \hline
Bitcrusher                    & B                 & b               & \multicolumn{6}{c|}{\begin{tabular}[c]{@{}c@{}}\underline{Bits}:\\ 1-16\end{tabular}}                                                                                             & \multicolumn{6}{c|}{\begin{tabular}[c]{@{}c@{}}\underline{Bitrate}: \\ 1-44100 hertz\end{tabular}}                                                                                                \\ \hline
Wah Wah                       & W                 & w               & \multicolumn{3}{c|}{\begin{tabular}[c]{@{}c@{}}\underline{Damp}:\\ 0.1-1.0\end{tabular}} & \multicolumn{3}{c|}{\begin{tabular}[c]{@{}c@{}}\underline{MinFreq}:\\ 400-1000 hertz\end{tabular}} & \multicolumn{3}{c|}{\begin{tabular}[c]{@{}c@{}}\underline{MaxFreq}:\\ 2500-3500 hertz\end{tabular}} & \multicolumn{3}{c|}{\begin{tabular}[c]{@{}c@{}}\underline{WahWah Freq}:\\ 1000-3000 hertz\end{tabular}} \\ \hline
\end{tabular}
\caption{Lista de comandos}
\label{tab:efectos}
\end{table}