\subsubsection{Paquetes a instalar}
\label{subsec:instalar}
Para esta sección, se instaló la distribución Linux Mint 17.1 (basada en Ubuntu) en una máquina virtual, de modo de poder saber qué es necesario instalar en un sistema desde 0 para poder correr el \textbf{TP}.

\subsubsection*{Compilar TP}
Para poder compilar el \textbf{TP} mediante el comando \textit{Make}, se necesitan los paquetes: 
\begin{itemize}
 \item \textbf{libsndfile1-dev} (librería libsndfile)
 \item \textbf{build-essential} (librería stdio.h)
 \item \textbf{nasm}
\end{itemize}

\subsubsection*{Interfaz gráfica}
Para poder ejecutar la interfaz gráfica:
\begin{itemize}
 \item \textbf{python3}
 \item \textbf{python3-pyqt5} (bindings de Qt para python3)
 \item \textbf{python3-pyqt5.multimedia} (para poder reproducir archivos de audio desde la GUI)
\end{itemize}

\subsubsection*{Debug}
Para las herramientas de debug, es necesario instalar:
\begin{itemize}
 \item \textbf{valgrind}
 \item \textbf{kdbg}
 \item \textbf{kcachegrind}
 \item \textbf{graphviz libgraphviz-perl} (sólo para poder ver el Call Graph en KCacheGrind)
\end{itemize}

\subsubsection*{Comparación visual señales de audio}
Si se desea realizar la comparación visual de las señales de audio explicada en la sección \ref{subsec:chequeo-diferencias}, será necesario instalar el siguiente paquete:
\begin{itemize}
 \item \textbf{audacity}
\end{itemize}