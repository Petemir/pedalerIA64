\subsection{Vibrato}
\label{subsec:desarrollo-vibrato}

\subsubsection{Descripción}
\label{subsec:desarrollo-vibrato-desc}

Vibrato es un efecto que consiste en \textit{la variación periódica de la frecuencia de un sonido}\footnote{\url{https://es.wikipedia.org/wiki/Vibrato}}. Esta variación puede conseguirse utilizando un \textbf{LFO}, como en el efecto anterior, pero en este caso lo que oscilará es la frecuencia (\textbf{pitch}) del sonido, y no el delay de la señal (que además, en el vibrato, el delay toma un valor mucho menor, 0-3ms). 

Para este efecto, se utilizó un buffer circular (\textit{dataBuffEffect}). El \textbf{LFO} toma los mismos parámetros que en \fullref{subsec:desarrollo-flanger}, el argumento \textbf{mod} del efecto (que simboliza la frecuencia de modulación) y el índice de la muestra actual. La parte entera de este resultado es el índice de la muestra que se utiliza en la señal húmeda, mientras que la parte fraccionaria se usa para realizar una interpolación entre la mencionada muestra y la anterior.

Al igual que para el caso del \fullref{subsec:desarrollo-flanger}, la rutina en \textbf{ASM} se encuentra dividida en dos archivos, \textbf{vibrato\_index\_calc.asm} y \textbf{vibrato.asm}.

\vspace{\baselineskip}

Los resultados de la comparación entre las versiones en \textbf{C} y \textbf{ASM} de este algoritmo se verán en la sección \fullref{subsec:resultados-vibrato}, y el análisis en \fullref{subsec:analisis-vibrato}.

\subsubsection{Pseudocódigo}
\label{subsec:desarrollo-vibrato-code}

\lstset{language=C}
\begin{lstlisting}[frame=single]
Argumentos = mod, depth
depth = redondear(depth*archivoEntrada.samplerate)
delay = depth
mod = mod/archivoEntrada.samplerate
Para cada muestra
  mod_actual = sen(mod*2*PI*indice_actual)
  tap = 1+delay+depth*mod_actual
  indice_muestra_humeda = floor(tap)
  frac = tap - indice_muestra_humeda

  dataBuffOut.canalDerecho =  dataBuffIn.muestra_humeda*frac + 
			      dataBuffIn.(muestra_humeda-1)*(1-frac)
  dataBuffOut.canalIzquierdo = dataBuffIn.muestraCicloActual
\end{lstlisting}

\subsubsection{Comando}
\label{subsec:desarrollo-vibrato-call}

\underline{\textbf{C}}:
\begin{center}
 \textit{./main INFILE OUTFILE -v depth mod}
\end{center}

\underline{\textbf{ASM}}:
\begin{center}
 \textit{./main INFILE OUTFILE -V depth mod}
\end{center}

\begin{itemize}
 \item \textit{depth}: argumento con rango entre 0.000 y 0.003s. Es el delay de la señal de entrada.
 \item \textit{mod}: argumento con rango entre 0.10 y 5.00Hz. Es la frecuencia de modulación del efecto.
\end{itemize}
 
