\subsection{Bitcrusher}
\label{subsec:desarrollo-bitcrusher}

\subsubsection{Descripción}
\label{subsec:desarrollo-bitcrusher-desc}


\vspace{\baselineskip}

Los resultados de la comparación entre las versiones en \textbf{C} y \textbf{ASM} de este algoritmo se verán en la sección \fullref{subsec:resultados-bitcrusher}, y el análisis en \fullref{subsec:analisis-bitcrusher}.

\subsubsection{Pseudocódigo}
\label{subsec:desarrollo-bitcrusher-code}

\lstset{language=C}
\begin{lstlisting}[frame=single]
\end{lstlisting}

\subsubsection{Comando}
\label{subsec:desarrollo-bitcrusher-call}

\underline{\textbf{C}}:
\begin{center}
 \textit{./main INFILE OUTFILE -}
\end{center}

\underline{\textbf{ASM}}:
\begin{center}
 \textit{./main INFILE OUTFILE -}
\end{center}

\begin{itemize}
 \item \textit{}: argumento con rango entre .
 \item \textit{}: argumento con rango entre .
\end{itemize}
 
 
