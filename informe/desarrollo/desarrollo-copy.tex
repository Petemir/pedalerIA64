\subsection{Copy}
\label{subsec:desarrollo-copy}

\subsubsection{Descripción}
\label{subsec:desarrollo-copy-desc}

No es un efecto en sí, pero fue desarrollado como prueba de concepto para el preinforme, como método para verificar que se estuviera usando bien la API de libsndfile (\ref{subsec:libsndfile-api}), la convención de llamado de funciones de ASM desde C, entre otras cosas.

\begin{center}
\fbox{\begin{minipage}{32em}
\underline{Nota}: en el preinforme, este algoritmo utilizaba doubles en vez de floats.
\end{minipage}}
\end{center}

Los resultados de la comparación entre las versiones en \textbf{C} y \textbf{ASM} de este algoritmo se verán en la sección \fullref{subsec:resultados-copy}, y el análisis en \fullref{subsec:analisis-copy}.

\subsubsection{Pseudocódigo}
\label{subsec:desarrollo-copy-code}

No se adjunta el pseudocódigo para este algoritmo por no aportar nada, pues es literalmente grabar en la salida lo que entra.

\subsubsection{Comando}
\label{subsec:desarrollo-copy-call}

\underline{\textbf{C}}:
\begin{center}
 \textit{./main INFILE OUTFILE -c}
\end{center}

\underline{\textbf{ASM}}:
\begin{center}
 \textit{./main INFILE OUTFILE -C}
\end{center}