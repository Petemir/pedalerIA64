\section{Resultados}
\label{sec:resultados}

Para generar un conjunto de archivos de salida que pudieran ser usados para la presente sección, se utilizó el archivo \textbf{generatorOutputExamples.py}, que para 4 de los archivos de audio de entrada en \textit{inputExamples/} (\textit{guitar.wav, gibson.wav, beirut.wav, DiMarzio.wav}) aplica todos los efectos (generando un archivo de salida por cada uno) con valores al azar en los argumentos (dentro de los rangos determinados). Este archivo genera la lista de comandos (que se halla en \textbf{generatorOutputExamples.sh}), y que es el que finalmente se usa para correr el programa.
 
Cada efecto se ejecutó 100 veces, para que la librería \textit{tiempo.h} pudiera tener una cantidad aceptable de iteraciones sobre la cual evaluar la cantidad de ciclos de procesador utilizados en cada algoritmo. Se incluyen en cada sección los comandos utilizados para generar esos resultados (aunque se omitieron los directorios que formaban parte del path absoluto de cada uno).

\subsection{Copy}
\label{subsec:resultados-copy} 

\subsection{Delay}
\label{subsec:resultados-delay}

\begin{center}
\begin{tikzpicture}
\begin{axis}[
    every axis plot post/.style={/pgf/number format/fixed},
    ybar=4pt,
    bar width=12pt,
    x=3cm,
    ymin=0,
    axis on top,
    %ymax=12,
    xtick=data,
    enlarge x limits=0.2,
    symbolic x coords={Guitar,Gibson,Beirut, DiMarzio},
    legend style={at={(0.5,1.0)}, anchor=north,legend columns=1},   
    %restrict y to domain*=0:14, % Cut values off at 14
    visualization depends on=rawy\as\rawy, % Save the unclipped values
    after end axis/.code={ % Draw line indicating break
      \draw [ultra thick, white, decoration={snake, amplitude=1pt}, decorate] (rel axis cs:0,1.05) -- (rel axis cs:1,1.05);
    },
    nodes near coords={%
            \pgfmathprintnumber{\rawy}% Print unclipped values
        },
    axis lines*=left,
    clip=false
    ]
\addplot coordinates {(Guitar.wav,13438043) (Gibson,5077729) (Beirut,26610728) (DiMarzio, 71669712)};
\addplot coordinates {(Guitar.wav,2834073) (Gibson,848546) (Beirut,5151332) (DiMarzio, 13861582)};
\legend{C, ASM}
\end{axis}
\end{tikzpicture}
\end{center}

\lstset{language=bash}
\begin{lstlisting}[frame=single]
./main guitar.wav guitar-delay-c-100iter-1.1-0.95.wav 100 -d 1.1 0.95
./main guitar.wav guitar-delay-asm-100iter-1.1-0.95.wav 100 -D 1.1 0.95
./main gibson.wav gibson-delay-c-100iter-4.7-0.75.wav 100 -d 4.7 0.75
./main gibson.wav gibson-delay-asm-100iter-4.7-0.75.wav 100 -D 4.7 0.75
./main beirut.wav beirut-delay-c-100iter-0.8-0.84.wav 100 -d 0.8 0.84
./main beirut.wav beirut-delay-asm-100iter-0.8-0.84.wav 100 -D 0.8 0.84
./main DiMarzio.wav DiMarzio-delay-c-100iter-1.3-0.72.wav 100 -d 1.3 0.72
./main DiMarzio.wav DiMarzio-delay-asm-100iter-1.3-0.72.wav 
							100 -D 1.3 0.72
\end{lstlisting} 

\subsection{Flanger}
\label{subsec:resultados-flanger}

\begin{center}
\begin{tikzpicture}
\begin{axis}[
    every axis plot post/.style={/pgf/number format/fixed},
    ybar=4pt,
    bar width=12pt,
    x=3cm,
    ymin=0,
    axis on top,
    %ymax=12,
    xtick=data,
    enlarge x limits=0.2,
    symbolic x coords={Guitar,Gibson,Beirut, DiMarzio},
    legend style={at={(0.5,1.0)}, anchor=north,legend columns=1},   
    %restrict y to domain*=0:14, % Cut values off at 14
    visualization depends on=rawy\as\rawy, % Save the unclipped values
    after end axis/.code={ % Draw line indicating break
      \draw [ultra thick, white, decoration={snake, amplitude=1pt}, decorate] (rel axis cs:0,1.05) -- (rel axis cs:1,1.05);
    },
    nodes near coords={%
            \pgfmathprintnumber{\rawy}% Print unclipped values
        },
    axis lines*=left,
    clip=false
    ]
\addplot coordinates {(Guitar.wav,60924596) (Gibson,24826448) (Beirut,136644752) (DiMarzio, 371752672)};
\addplot coordinates {(Guitar.wav,96925120) (Gibson,36257624) (Beirut,214741360) (DiMarzio, 581458944)};
\legend{C, ASM}
\end{axis}
\end{tikzpicture}
\end{center}

\lstset{language=bash}
\begin{lstlisting}[frame=single]
./main guitar.wav guitar-flanger-c-100iter-0.011-0.4-0.73.wav 
                                    100 -f 0.011 0.4 0.73
./main guitar.wav guitar-flanger-asm-100iter-0.011-0.4-0.73.wav 
                                    100 -F 0.011 0.4 0.73
./main gibson.wav gibson-flanger-c-100iter-0.006-0.83-0.7.wav 
                                    100 -f 0.006 0.83 0.7
./main gibson.wav gibson-flanger-asm-100iter-0.006-0.83-0.7.wav 
                                    100 -F 0.006 0.83 0.7
./main beirut.wav beirut-flanger-c-100iter-0.001-0.56-0.68.wav 
                                    100 -f 0.001 0.56 0.68
./main beirut.wav beirut-flanger-asm-100iter-0.001-0.56-0.68.wav 
                                    100 -F 0.001 0.56 0.68
./main DiMarzio.wav DiMarzio-flanger-c-100iter-0.015-0.97-0.73.wav 
                                    100 -f 0.015 0.97 0.73
./main DiMarzio.wav DiMarzio-flanger-asm-100iter-0.015-0.97-0.73.wav 
                                    100 -F 0.015 0.97 0.73
\end{lstlisting} 

\subsection{Vibrato}
\label{subsec:resultados-vibrato} 

\begin{center}
\begin{tikzpicture}
\begin{axis}[
    every axis plot post/.style={/pgf/number format/fixed},
    ybar=4pt,
    bar width=12pt,
    x=3cm,
    ymin=0,
    axis on top,
    %ymax=12,
    xtick=data,
    enlarge x limits=0.2,
    symbolic x coords={Guitar,Gibson,Beirut, DiMarzio},
    legend style={at={(0.5,1.0)}, anchor=north,legend columns=1},   
    %restrict y to domain*=0:14, % Cut values off at 14
    visualization depends on=rawy\as\rawy, % Save the unclipped values
    after end axis/.code={ % Draw line indicating break
      \draw [ultra thick, white, decoration={snake, amplitude=1pt}, decorate] (rel axis cs:0,1.05) -- (rel axis cs:1,1.05);
    },
    nodes near coords={%
            \pgfmathprintnumber{\rawy}% Print unclipped values
        },
    axis lines*=left,
    clip=false
    ]
\addplot coordinates {(Guitar.wav,54332088) (Gibson,22116198) (Beirut,121003160) (DiMarzio, 326084704)};
\addplot coordinates {(Guitar.wav,5137922) (Gibson,2162678) (Beirut,11273407) (DiMarzio, 31488804)};
\legend{C, ASM}
\end{axis}
\end{tikzpicture}
\end{center}

\lstset{language=bash}
\begin{lstlisting}[frame=single]
./main guitar.wav guitar-vibrato-c-100iter-0.002-3.77.wav 
					    100 -v 0.002 3.77
./main guitar.wav guitar-vibrato-asm-100iter-0.002-3.77.wav 
					    100 -V 0.002 3.77
./main gibson.wav gibson-vibrato-c-100iter-0.003-2.89.wav 
					    100 -v 0.003 2.89
./main gibson.wav gibson-vibrato-asm-100iter-0.003-2.89.wav 
					    100 -V 0.003 2.89
./main beirut.wav beirut-vibrato-c-100iter-0.002-4.39.wav 
					    100 -v 0.002 4.39
./main beirut.wav beirut-vibrato-asm-100iter-0.002-4.39.wav 
					    100 -V 0.002 4.39
./main DiMarzio.wav DiMarzio-vibrato-c-100iter-0.0-4.23.wav 
					    100 -v 0.0 4.23
./main DiMarzio.wav DiMarzio-vibrato-asm-100iter-0.0-4.23.wav 
					    100 -V 0.0 4.23

\end{lstlisting} 

\subsection{Bitcrusher}
\label{subsec:resultados-bitcrusher} 

\begin{figure}[H]
\begin{center}
\begin{tikzpicture}
\begin{axis}[
    every axis plot post/.style={/pgf/number format/fixed},
    ybar=4pt,
    bar width=12pt,
    x=3cm,
    ymin=0,
    axis on top,
    %ymax=12,
    xtick=data,
    enlarge x limits=0.2,
    symbolic x coords={Guitar,Gibson,Beirut, DiMarzio},
    legend style={at={(0.5,1.0)}, anchor=north,legend columns=1},   
    %restrict y to domain*=0:14, % Cut values off at 14
    visualization depends on=rawy\as\rawy, % Save the unclipped values
    after end axis/.code={ % Draw line indicating break
      \draw [ultra thick, white, decoration={snake, amplitude=1pt}, decorate] (rel axis cs:0,1.05) -- (rel axis cs:1,1.05);
    },
    nodes near coords={%
            \pgfmathprintnumber{\rawy}% Print unclipped values
        },
    every node near coord/.append style={font=\tiny, inner sep=1pt},        
    axis lines*=left,
    clip=false
    ]
\addplot coordinates {(Guitar.wav,13478947) (Gibson,4030649) (Beirut,30549640) (DiMarzio, 84020784)};
\addplot coordinates {(Guitar.wav,2505268) (Gibson,661215) (Beirut,4749857) (DiMarzio, 14452590)};
\legend{C, ASM}
\end{axis}
\end{tikzpicture}
\caption{Bitcrusher Libreria Tiempo.h}
\label{fig:resultados-bitcrusher-tiempo}
\end{center}
\end{figure}


\begin{figure}[H]
\begin{center}
\begin{tikzpicture}
\begin{axis}[
    every axis plot post/.style={/pgf/number format/fixed},
    ybar=4pt,
    bar width=12pt,
    x=3cm,
    ymin=0,
    axis on top,
    %ymax=12,
    xtick=data,
    enlarge x limits=0.2,
    symbolic x coords={Guitar,Gibson,Beirut, DiMarzio},
    legend style={at={(0.5,1.0)}, anchor=north,legend columns=1},   
    %restrict y to domain*=0:14, % Cut values off at 14
    visualization depends on=rawy\as\rawy, % Save the unclipped values
    after end axis/.code={ % Draw line indicating break
      \draw [ultra thick, white, decoration={snake, amplitude=1pt}, decorate] (rel axis cs:0,1.05) -- (rel axis cs:1,1.05);
    },
    nodes near coords={%
            \pgfmathprintnumber{\rawy}% Print unclipped values
        },
    every node near coord/.append style={font=\tiny, inner sep=1pt},
    axis lines*=left,
    clip=false
    ]
\addplot coordinates {(Guitar.wav,23322057) (Gibson,7676128) (Beirut,49948174) (DiMarzio,163708326)};
\addplot coordinates {(Guitar.wav,10458554) (Gibson,4113654) (Beirut,22976589) (DiMarzio,89227240)};
\legend{C, ASM}
\end{axis}
\end{tikzpicture}
\caption{Bitcrusher Callgrind}
\label{fig:resultados-Bitcrusher-callgrind}
\end{center}
\end{figure}

%\lstset{language=bash}
%\begin{lstlisting}[frame=single]
%./main guitar.wav guitar-bitcrusher-c-100iter-9-10233.wav
%					100 -b 9 10233
%./main guitar.wav guitar-bitcrusher-asm-100iter-9-10233.wav
%					100 -B 9 10233
%./main gibson.wav gibson-bitcrusher-c-100iter-12-10024.wav 
%					100 -b 12 10024
%./main gibson.wav gibson-bitcrusher-asm-100iter-12-10024.wav 
%					100 -B 12 10024
%./main beirut.wav beirut-bitcrusher-c-100iter-14-6752.wav 
%					100 -b 14 6752
%./main beirut.wav beirut-bitcrusher-asm-100iter-14-6752.wav 
%					100 -B 14 6752
%./main DiMarzio.wav DiMarzio-bitcrusher-c-100iter-12-7888.wav 
%					100 -b 12 7888
%./main DiMarzio.wav DiMarzio-bitcrusher-asm-100iter-12-7888.wav 
%					100 -B 12 7888
%\end{lstlisting} 

\subsection{wahwah}
\label{subsec:resultados-wahwah} 

\begin{center}
\begin{tikzpicture}
\begin{axis}[
    every axis plot post/.style={/pgf/number format/fixed},
    ybar=4pt,
    bar width=12pt,
    x=3cm,
    ymin=0,
    axis on top,
    %ymax=12,
    xtick=data,
    enlarge x limits=0.2,
    symbolic x coords={Guitar,Gibson,Beirut, DiMarzio},
    legend style={at={(0.5,1.0)}, anchor=north,legend columns=1},   
    %restrict y to domain*=0:14, % Cut values off at 14
    visualization depends on=rawy\as\rawy, % Save the unclipped values
    after end axis/.code={ % Draw line indicating break
      \draw [ultra thick, white, decoration={snake, amplitude=1pt}, decorate] (rel axis cs:0,1.05) -- (rel axis cs:1,1.05);
    },
    nodes near coords={%
            \pgfmathprintnumber{\rawy}% Print unclipped values
        },
    axis lines*=left,
    clip=false
    ]
\addplot coordinates {(Guitar.wav,79632992) (Gibson,33243144) (Beirut,173768288) (DiMarzio, 462280256)};
\addplot coordinates {(Guitar.wav,1190239) (Gibson,453681) (Beirut,2542485) (DiMarzio, 6787702)};
\legend{C, ASM}
\end{axis}
\end{tikzpicture}
\end{center}

\lstset{language=bash}
\begin{lstlisting}[frame=single]
./main guitar.wav guitar-wahwah-c-100iter-0.08-424-2853-2931.wav 
					100 -w 0.08 424 2853 2931
./main guitar.wav guitar-wahwah-asm-100iter-0.08-424-2853-2931.wav 
					100 -W 0.08 424 2853 2931
./main gibson.wav gibson-wahwah-c-100iter-0.06-541-3003-2937.wav 
					100 -w 0.06 541 3003 2937
./main gibson.wav gibson-wahwah-asm-100iter-0.06-541-3003-2937.wav 
					100 -W 0.06 541 3003 2937
./main beirut.wav beirut-wahwah-c-100iter-0.07-690-2532-1963.wav 
					100 -w 0.07 690 2532 1963
./main beirut.wav beirut-wahwah-asm-100iter-0.07-690-2532-1963.wav 
					100 -W 0.07 690 2532 1963
./main DiMarzio.wav DiMarzio-wahwah-c-100iter-0.1-554-2689-2109.wav 
					100 -w 0.1 554 2689 2109
./main DiMarzio.wav DiMarzio-wahwah-asm-100iter-0.1-554-2689-2109.wav 
					100 -W 0.1 554 2689 2109
\end{lstlisting} 